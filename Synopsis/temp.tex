\subsection{Results}
\begin{enumerate}
\item Database for analysis
	\begin{itemize}
	\item 200 guassian distribution generated random sequences was subjected to MCMC simulation.
	\item 30 sequences have Foldicity > 0.4 and 146 sequences did not fold at all.
	\end{itemize}
\item Test of metastability of folded states
	\begin{itemize}
	\item 146 unfolding sequences are tested to determine whether any of them end up in a single metastable state/local minima.
	\item Each sequence was run for 10 independant folding simulations. Conformation with lowest energy was compared for all the 10 simulations for each of these 146 sequences. None of these sequences folded into same local energy minimum and thus the present model does not support the metastable native state model and is in accordance with results of Honeycutt and Thirumalai, 1992.
	\end{itemize}
\item Comparison of lower part of energy spectrum of the ensemble with CSA chains
	\begin{itemize}
	\item Only CSA chains are considered instead of the ensemble of self-avoiding chains.
	\item Sample energy spectra of 10 random sequences showed that if teh discrete part of energy spectrum is sparse for CSE chain then it is likely to be sparse for all chain, i.e., $X$ is strongly co-related to $X^{cse}$
	\end{itemize}
\item Relation between foldicity and energy spectrum
	\begin{itemize}
	\item Folding is more associated with sequence having pronounced energy global minimum.
	\item $<X^{csa}(T)>$ for non folding sequence is higher than that of folding sequences.
	\item $(E_1 - E_0)$ and $T(X^{csa}=0.8)$ are strongly correlated and determines whether a not a given sequence is a folding sequence.
	\end{itemize}
\item Association between foldicity and conformation and native state
	\begin{itemize}
	\item Order of contact : Absolute difference between the indices of two monomers in the chain
	\item Contacts with less order are more likely to occur as compared to contacts with higher order. This resulted is in accordance with Wetlaufer, 1973.
	\end{itemize}
\end{enumerate}